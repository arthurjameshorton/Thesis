\chapter{Response vs. the subleading jet cut at EM+GSC using $\gamma$+jet events in 2016}
%\label{App:GJetEM2016}

\begin{figure}[!ht]
\centering
\begin{subfigure}{.5\textwidth}
\centering
\scalebox{0.39}{
  \includegraphics{plots/Chap5/RadiationPlots_PhotonEM2016/J2/ResponseVsJ2_25-45.eps}
}
\caption{25-45 GeV}
\end{subfigure}%
\begin{subfigure}{.5\textwidth}
\centering
\scalebox{0.39}{
  \includegraphics{plots/Chap5/RadiationPlots_PhotonEM2016/J2/ResponseVsJ2_45-65.eps}
}
\caption{45-65 GeV}
\end{subfigure}
  \caption[Response as a function of the subleading jet cut, 25-45 and 45-65 GeV bins]
  {\small Response as a function of the fraction of the photon energy contained in the subleading jet (J2) in two different $p_{\mathrm T}$ bins along with the distribution of events plotted against the same variable.  The excess of events with a recorded subleading jet $p_{\mathrm T}$ being 0\% of the reference $p_{\mathrm T}$ is caused by the subleading jet being below the jet reconstruction threshold.  All selection cuts listed in Sec.~\ref{Sec:SelectionCriteria} have been applied except for the subleading jet cut and the $\Delta\phi$ cut.  }
  \label{plot:GJetEMJ225-45_2016App}
\end{figure}

\begin{figure}[!ht]
  \centering
  \begin{subfigure}{.5\textwidth}
    \centering
    \scalebox{0.39}{
      \includegraphics{plots/Chap5/RadiationPlots_PhotonEM2016/J2/ResponseVsJ2_65-85.eps}
    }
    \caption{65-85 GeV}
  \end{subfigure}%
  \begin{subfigure}{.5\textwidth}
    \centering
    \scalebox{0.39}{
      \includegraphics{plots/Chap5/RadiationPlots_PhotonEM2016/J2/ResponseVsJ2_85-105.eps}
    }
    \caption{85-105 GeV}
  \end{subfigure}
  \caption[Response as a function of the subleading jet cut, 65-85 and 85-105 GeV bins]
  {\small Response as a function of the fraction of the photon energy contained in the subleading jet (J2) in two different $p_{\mathrm T}$ bins along with the distribution of events plotted against the same variable.  The excess of events with a recorded subleading jet $p_{\mathrm T}$ being 0\% of the reference $p_{\mathrm T}$ is caused by the subleading jet being below the jet reconstruction threshold.  All selection cuts listed in Sec.~\ref{Sec:SelectionCriteria} have been applied except for the subleading jet cut and the $\Delta\phi$ cut.  }
  \label{plot:GJetEMJ265-85_2016App}
\end{figure}

\begin{figure}[!ht]
  \centering
  \begin{subfigure}{.5\textwidth}
    \centering
    \scalebox{0.39}{
      \includegraphics{plots/Chap5/RadiationPlots_PhotonEM2016/J2/ResponseVsJ2_105-125.eps}
    }
    \caption{105-125 GeV}
  \end{subfigure}%
  \begin{subfigure}{.5\textwidth}
    \centering
    \scalebox{0.39}{
      \includegraphics{plots/Chap5/RadiationPlots_PhotonEM2016/J2/ResponseVsJ2_125-160.eps}
    }
    \caption{125-160 GeV}
  \end{subfigure}
  \caption[Response as a function of the subleading jet cut, 105-125 and 125-160 GeV bins]
  {\small Response as a function of the fraction of the photon energy contained in the subleading jet (J2) in two different $p_{\mathrm T}$ bins along with the distribution of events plotted against the same variable.  The excess of events with a recorded subleading jet $p_{\mathrm T}$ being 0\% of the reference $p_{\mathrm T}$ is caused by the subleading jet being below the jet reconstruction threshold.  All selection cuts listed in Sec.~\ref{Sec:SelectionCriteria} have been applied except for the subleading jet cut and the $\Delta\phi$ cut.  }
  \label{plot:GJetEMJ2105-125_2016App}
\end{figure}

\begin{figure}[!ht]
  \centering
  \begin{subfigure}{.5\textwidth}
    \centering
    \scalebox{0.39}{
      \includegraphics{plots/Chap5/RadiationPlots_PhotonEM2016/J2/ResponseVsJ2_160-210.eps}
    }
    \caption{160-210 GeV}
  \end{subfigure}%
  \begin{subfigure}{.5\textwidth}
    \centering
    \scalebox{0.39}{
      \includegraphics{plots/Chap5/RadiationPlots_PhotonEM2016/J2/ResponseVsJ2_210-260.eps}
    }
    \caption{210-260 GeV}
  \end{subfigure}
   \caption[Response as a function of the subleading jet cut, 160-210 and 210-260 GeV bins]
  {\small Response as a function of the fraction of the photon energy contained in the subleading jet (J2) in two different $p_{\mathrm T}$ bins along with the distribution of events plotted against the same variable.  The excess of events with a recorded subleading jet $p_{\mathrm T}$ being 0\% of the reference $p_{\mathrm T}$ is caused by the subleading jet being below the jet reconstruction threshold.  All selection cuts listed in Sec.~\ref{Sec:SelectionCriteria} have been applied except for the subleading jet cut and the $\Delta\phi$ cut.  }
  \label{plot:GJetEMJ2160-210_2016App}
\end{figure}

\begin{figure}[!ht]
  \centering
  \begin{subfigure}{.5\textwidth}
    \centering
    \scalebox{0.39}{
      \includegraphics{plots/Chap5/RadiationPlots_PhotonEM2016/J2/ResponseVsJ2_260-310.eps}
    }
    \caption{260-310 GeV}
  \end{subfigure}%
  \begin{subfigure}{.5\textwidth}
    \centering
    \scalebox{0.39}{
      \includegraphics{plots/Chap5/RadiationPlots_PhotonEM2016/J2/ResponseVsJ2_310-400.eps}
    }
    \caption{310-400 GeV}
  \end{subfigure}
  \caption[Response as a function of the subleading jet cut, 260-310 and 310-200 GeV bins]
  {\small Response as a function of the fraction of the photon energy contained in the subleading jet (J2) in two different $p_{\mathrm T}$ bins along with the distribution of events plotted against the same variable.  All selection cuts listed in Sec.~\ref{Sec:SelectionCriteria} have been applied except for the subleading jet cut and the $\Delta\phi$ cut.  }
  \label{plot:GJetEMJ2260-310_2016App}
\end{figure}

\begin{figure}[!ht]
  \centering
  \begin{subfigure}{.5\textwidth}
    \centering
    \scalebox{0.39}{
      \includegraphics{plots/Chap5/RadiationPlots_PhotonEM2016/J2/ResponseVsJ2_400-500.eps}
    }
    \caption{400-500 GeV}
  \end{subfigure}%
  \begin{subfigure}{.5\textwidth}
    \centering
    \scalebox{0.39}{
      \includegraphics{plots/Chap5/RadiationPlots_PhotonEM2016/J2/ResponseVsJ2_500-600.eps}
    }
    \caption{500-600 GeV}
  \end{subfigure}
  \caption[Response as a function of the subleading jet cut, 400-500 and 500-600 GeV bins]
  {\small Response as a function of the fraction of the photon energy contained in the subleading jet (J2) in two different $p_{\mathrm T}$ bins along with the distribution of events plotted against the same variable.  All selection cuts listed in Sec.~\ref{Sec:SelectionCriteria} have been applied except for the subleading jet cut and the $\Delta\phi$ cut.  }
  \label{plot:GJetEMJ2400-500_2016App}
\end{figure}

\begin{figure}[!ht]
  \centering
  \begin{subfigure}{.5\textwidth}
    \centering
    \scalebox{0.39}{
      \includegraphics{plots/Chap5/RadiationPlots_PhotonEM2016/J2/ResponseVsJ2_600-800.eps}
    }
    \caption{600-800 GeV}
  \end{subfigure}%
  \begin{subfigure}{.5\textwidth}
    \centering
    \scalebox{0.39}{
      \includegraphics{plots/Chap5/RadiationPlots_PhotonEM2016/J2/ResponseVsJ2_800-1000.eps}
    }
    \caption{800-1000 GeV}
  \end{subfigure}
  \caption[Response as a function of the subleading jet cut, 600-800 and 800-1000 GeV bins]
  {\small Response as a function of the fraction of the photon energy contained in the subleading jet (J2) in two different $p_{\mathrm T}$ bins along with the distribution of event
s plotted against the same variable.  All selection cuts listed in Sec.~\ref{Sec:SelectionCriteria} have been applied except for the subleading jet cut and the $\Delta\phi$ cut.  }
  \label{plot:GJetEMJ2600-800_2016App}
\end{figure}

