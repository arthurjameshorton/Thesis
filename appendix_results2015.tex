\chapter{2015 Results}
\label{App:2015}

\begin{figure}[!ht]
  \begin{center}
  \scalebox{0.6}{
    \includegraphics{plots/Appendix/2015/Photon/EM/Photon_Response_EM.eps}
  }
  \end{center}
  \caption[EM scale response using $\gamma$+jet in 2015]
  {\small The EM+GSC scale response measured using the MPF technique in $\gamma$+jet events.  Shown in black is the response measured using data from the 2015 dataset, in red is the nominal Monte Carlo sample (Powheg+Pythia) and in clue is an additional Monte Carlo sample for comparison.  The lower inset shows the data to Monte Carlo ratio, with the colour of the points corresponding to the simulated sample used in the ratio.  }
  \label{plot:GammaJetEM2015App}
\end{figure}


\begin{figure}[!ht]
\captionsetup[subfigure]{labelformat=empty}
 \begin{center}
   \begin{subfigure}{0.55\textwidth}
     \hspace{-3cm}
     \includegraphics[width=2.3\linewidth, angle=90]{plots/Appendix/2015/Photon/EM/SystsTwoSided.eps}
   \end{subfigure}
   \begin{subfigure}{0.55\textwidth}
     \hspace{-3cm}
     \includegraphics[width=1.5\linewidth]{plots/Appendix/2015/Photon/EM/Legend.eps}
   \end{subfigure}
 \end{center}
 \caption[Uncertainty on the EM+GSC scale response measurement using $\gamma$+jet]
 {\small The total uncertainty (both statistical and systematic) on the measurement of the relative EM+GSC scale response between data and MC using $\gamma$+jet events from 2015.  It is broken down into the various uncertainty sources that go into the total uncertainty.  }
 \label{Fig:GammaJetSystsEM2015}
\end{figure}

\begin{figure}[!ht]
  \begin{center}
  \scalebox{0.6}{
    \includegraphics{plots/Appendix/2015/Photon/LC/Photon_Response_LC.eps}
  }
  \end{center}
  \caption[LC scale response using $\gamma$+jet in 2015]
  {\small The LC+GSC scale response measured using the MPF technique in $\gamma$+jet events.  Shown in black is the response measured using data from the 2015 dataset, in red is the nominal Monte Carlo sample (Powheg+Pythia) and in clue is an additional Monte Carlo sample for comparison.  The lower inset shows the data to Monte Carlo ratio, with the colour of the points corresponding to the simulated sample used in the ratio.  }
  \label{plot:GammaJetLC2015App}
\end{figure}


\begin{figure}[!ht]
\captionsetup[subfigure]{labelformat=empty}
 \begin{center}
   \begin{subfigure}{0.55\textwidth}
     \hspace{-3cm}
     \includegraphics[width=2.3\linewidth, angle=90]{plots/Appendix/2015/Photon/LC/SystsTwoSided.eps}
   \end{subfigure}
   \begin{subfigure}{0.55\textwidth}
     \hspace{-3cm}
     \includegraphics[width=1.5\linewidth]{plots/Appendix/2015/Photon/LC/Legend.eps}
   \end{subfigure}
 \end{center}
 \caption[Uncertainty on the LC+GSC scale response measurement using $\gamma$+jet]
 {\small The total uncertainty (both statistical and systematic) on the measurement of the relative LC+GSC scale response between data and MC using $\gamma$+jet events from 2015.  It is broken down into the various uncertainty sources that go into the total uncertainty.  }
 \label{Fig:GammaJetSystsLC2015}
\end{figure}

%   Z jet
%   EM scale
\begin{figure}[!ht]
  \begin{center}
  \scalebox{0.6}{
    \includegraphics{plots/Appendix/2015/Z/EM/Z_Response_EM.eps}
  }
  \end{center}
  \caption[EM scale response using Z+jet in 2015]
  {\small The EM+GSC scale response measured using the MPF technique in Z+jet events.  Shown in black is the response measured using data from the 2015 dataset, in red is the nominal Monte Carlo sample (Powheg+Pythia) and in clue is an additional Monte Carlo sample for comparison.  The lower inset shows the data to Monte Carlo ratio, with the colour of the points corresponding to the simulated sample used in the ratio.  }
  \label{plot:ZJetEM2015App}
\end{figure}


\begin{figure}[!ht]
\captionsetup[subfigure]{labelformat=empty}
 \begin{center}
   \begin{subfigure}{0.55\textwidth}
     \hspace{-3cm}
     \includegraphics[width=2.3\linewidth, angle=90]{plots/Appendix/2015/Z/EM/SystsTwoSided.eps}
   \end{subfigure}
   \begin{subfigure}{0.55\textwidth}
     \hspace{-3cm}
     \includegraphics[width=1.5\linewidth]{plots/Appendix/2015/Z/EM/Legend.eps}
   \end{subfigure}
 \end{center}
 \caption[Uncertainty on the EM+GSC scale response measurement using Z+jet]
 {\small The total uncertainty (both statistical and systematic) on the measurement of the relative EM+GSC scale response between data and MC using Z+jet events from 2015.  I
t is broken down into the various uncertainty sources that go into the total uncertainty.  }
 \label{Fig:ZJetSystsEM2015}
\end{figure}


%   Z jet
%   LC scale
\begin{figure}[!ht]
  \begin{center}
  \scalebox{0.6}{
    \includegraphics{plots/Appendix/2015/Z/LC/Z_Response_LC.eps}
  }
  \end{center}
  \caption[LC scale response using Z+jet in 2015]
  {\small The LC+GSC scale response measured using the MPF technique in Z+jet events.  Shown in black is the response measured using data from the 2015 dataset, in red is the nominal Monte Carlo sample (Powheg+Pythia) and in clue is an additional Monte Carlo sample for comparison.  The lower inset shows the data to Monte Carlo ratio, with the colour of the points corresponding to the simulated sample used in the ratio.  }
  \label{plot:ZJetLC2015App}
\end{figure}


\begin{figure}[!ht]
\captionsetup[subfigure]{labelformat=empty}
 \begin{center}
   \begin{subfigure}{0.55\textwidth}
     \hspace{-3cm}
     \includegraphics[width=2.3\linewidth, angle=90]{plots/Appendix/2015/Z/LC/SystsTwoSided.eps}
   \end{subfigure}
   \begin{subfigure}{0.55\textwidth}
     \hspace{-3cm}
     \includegraphics[width=1.5\linewidth]{plots/Appendix/2015/Z/LC/Legend.eps}
   \end{subfigure}
 \end{center}
 \caption[Uncertainty on the LC+GSC scale response measurement using Z+jet]
 {\small The total uncertainty (both statistical and systematic) on the measurement of the relative LC+GSC scale response between data and MC using Z+jet events from 2015.  I
t is broken down into the various uncertainty sources that go into the total uncertainty.  }
 \label{Fig:ZJetSystsLC2015}
\end{figure}



