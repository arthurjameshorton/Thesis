\chapter{2016 JES Results}
\label{App:2016}

\begin{figure}[!ht]
  \begin{center}
  \scalebox{1.0}{
    \includegraphics[angle=90]{plots/Appendix/GJet/EM/Photon_EM_Data_2016_ResponseVsPt_bins.eps}
  }
  \end{center}
  \caption[EM scale response distributions in data using $\gamma$+jet in 2016]
  {\small EM+GSC scale response distributions using $\gamma$+jet events in 2016 data.  Also shown in red is a Gaussian curve which has been sit using the procedure described in Sec.~\ref{Sec:MeasureJES}. }
  \label{plot:GJetEM2016DataDistsApp}
\end{figure}

\begin{figure}[!ht]
  \begin{center}
  \scalebox{1.0}{
    \includegraphics[angle=90]{plots/Appendix/GJet/EM/Photon_EM_Pythia_ResponseVsPt_bins.eps}
  }
  \end{center}
  \caption[EM scale response distributions in Pythia using $\gamma$+jet in 2016]
  {\small EM+GSC scale response distributions using $\gamma$+jet events using Pythia.  Also shown in red is a Gaussian curve which has been sit using the procedure described in Sec.~\ref{Sec:MeasureJES}. }
  \label{plot:GJetEM2016PythiaistsApp}
\end{figure}

\begin{figure}[!ht]
  \begin{center}
  \scalebox{1.0}{
    \includegraphics[angle=90]{plots/Appendix/GJet/EM/Photon_EM_Sherpa_ResponseVsPt_bins.eps}
  }
  \end{center}
  \caption[EM scale response distributions in Sherpa using $\gamma$+jet in 2016]
  {\small EM+GSC scale response distributions using $\gamma$+jet events in using Sherpa.  Also shown in red is a Gaussian curve which has been sit using the procedure described in Sec.~\ref{Sec:MeasureJES}. }
  \label{plot:GJetEM2016SherpaDistsApp}
\end{figure}

\begin{figure}[!ht]
  \begin{center}
  \scalebox{0.6}{
    \includegraphics{plots/Chap5/Photon_IssueFix/mpf_Ratio_EM_2016.eps}
  }
  \end{center}
  \caption[EM scale response using $\gamma$+jet in 2016]
  {\small The EM+GSC scale response measured using the MPF technique in $\gamma$+jet events.  Shown in black is the response measured using data from the 2016 dataset, in red is the nominal Monte Carlo sample (Powheg+Pythia) and in clue is an additional Monte Carlo sample for comparison.  The lower inset shows the data to Monte Carlo ratio, with the colour of the points corresponding to the simulated sample used in the ratio.  }
  \label{plot:GJetEM2016App}
\end{figure}

\begin{figure}[!ht]
\captionsetup[subfigure]{labelformat=empty}
 \begin{center}
   \begin{subfigure}{0.55\textwidth}
     \hspace{-3cm}
     \includegraphics[width=2.3\linewidth, angle=90]{plots/Chap5/Photon_IssueFix/SystsTwoSided.eps}
   \end{subfigure}
   \begin{subfigure}{0.55\textwidth}     \hspace{-3cm}
     \includegraphics[width=1.5\linewidth]{plots/Chap5/Photon_IssueFix/Legend.eps}
   \end{subfigure}
 \end{center}
 \caption[Uncertainty on the EM+GSC scale response measurement using $\gamma$+jet]
 {\small The total uncertainty (both statistical and systematic) on the measurement of the relative EM+GSC scale response between data and MC using $\gamma$+jet events.  It is broken down into the various uncertainty sources that go into the total uncertainty.  }
 \label{Fig:GJetSystsEM2016}
\end{figure}

\begin{figure}[!ht]
  \begin{center}
  \scalebox{0.6}{
    \includegraphics{plots/Appendix/ZJet/EM/mpf_Ratio.eps}
  }
  \end{center}
  \caption[EM scale response using Z+jet in 2016]
  {\small The EM+GSC scale response measured using the MPF technique in Z+jet events.  Shown in black is the response measured using data from the 2016 dataset, in red is the nominal Monte Carlo sample (Powheg+Pythia) and in clue is an additional Monte Carlo sample for comparison.  The lower inset shows the data to Monte Carlo ratio, with the colour of the points corresponding to the simulated sample used in the ratio.  }
  \label{plot:ZJetEM2016App}
\end{figure}

\begin{figure}[!ht]
\captionsetup[subfigure]{labelformat=empty}
 \begin{center}
   \begin{subfigure}{0.55\textwidth}
     \hspace{-3cm}
     \includegraphics[width=2.3\linewidth, angle=90]{plots/Chap5/Z/EM/SystsTwoSided.eps}
   \end{subfigure}
   \begin{subfigure}{0.55\textwidth}     \hspace{-3cm}
     \includegraphics[width=1.5\linewidth]{plots/Chap5/Z/EM/Legend.eps}
   \end{subfigure}
 \end{center}
 \caption[Uncertainty on the EM+GSC scale response measurement using Z+jet]
 {\small The total uncertainty (both statistical and systematic) on the measurement of the relative EM+GSC scale response between data and MC using Z+jet events.  It is broken down
into the various uncertainty sources that go into the total uncertainty.  }
 \label{Fig:ZJetSystsEM2016}
\end{figure}

\begin{figure}[!ht]
  \begin{center}
  \scalebox{0.6}{
    \includegraphics{plots/Chap5/Photon/LC/mpf_Ratio.eps}
  }
  \end{center}
  \caption[LC scale response using $\gamma$+jet in 2016]
  {\small The LC+GSC scale response measured using the MPF technique in $\gamma$+jet events.  Shown in black is the response measured using data from the 2016 dataset, in red is the nominal Monte Carlo sample (Powheg+Pythia) and in clue is an additional Monte Carlo sample for comparison.  The lower inset shows the data to Monte Carlo ratio, with the colour of the points corresponding to the simulated sample used in the ratio.  }
  \label{plot:GJetLC2016App}
\end{figure}

\begin{figure}[!ht]
\captionsetup[subfigure]{labelformat=empty}
 \begin{center}
   \begin{subfigure}{0.55\textwidth}
     \hspace{-3cm}
     \includegraphics[width=2.3\linewidth, angle=90]{plots/Chap5/Photon/LC/SystsTwoSided.eps}
   \end{subfigure}
   \begin{subfigure}{0.55\textwidth}     \hspace{-3cm}
     \includegraphics[width=1.5\linewidth]{plots/Chap5/Photon/LC/Legend.eps}
   \end{subfigure}
 \end{center}
 \caption[Uncertainty on the LC+GSC scale response measurement using $\gamma$+jet]
 {\small The total uncertainty (both statistical and systematic) on the measurement of the relative LC+GSC scale response between data and MC using $\gamma$+jet events.  It is broken down into the various uncertainty sources that go into the total uncertainty.  }
 \label{Fig:GJetSystsLC2016}
\end{figure}

\begin{figure}[!ht]
  \begin{center}
  \scalebox{0.6}{
    \includegraphics{plots/Appendix/ZJet/LC/mpf_Ratio.eps}
  }
  \end{center}
  \caption[LC scale response using Z+jet in 2016]
  {\small The LC+GSC scale response measured using the MPF technique in Z+jet events.  Shown in black is the response measured using data from the 2016 dataset, in red is the nominal Monte Carlo sample (Powheg+Pythia) and in clue is an additional Monte Carlo sample for comparison.  The lower inset shows the data to Monte Carlo ratio, with the colour of the points corresponding to the simulated sample used in the ratio.  }
  \label{plot:ZJetLC2016App}
\end{figure}

\begin{figure}[!ht]
\captionsetup[subfigure]{labelformat=empty}
 \begin{center}
   \begin{subfigure}{0.55\textwidth}
     \hspace{-3cm}
     \includegraphics[width=2.3\linewidth, angle=90]{plots/Chap5/Z/LC/SystsTwoSided.eps}
   \end{subfigure}
   \begin{subfigure}{0.55\textwidth}     \hspace{-3cm}
     \includegraphics[width=1.5\linewidth]{plots/Chap5/Z/LC/Legend.eps}
   \end{subfigure}
 \end{center}
 \caption[Uncertainty on the LC+GSC scale response measurement using Z+jet]
 {\small The total uncertainty (both statistical and systematic) on the measurement of the relative LC+GSC scale response between data and MC using Z+jet events.  It is broken down
into the various uncertainty sources that go into the total uncertainty.  }
 \label{Fig:ZJetSystsLC2016}
\end{figure}

