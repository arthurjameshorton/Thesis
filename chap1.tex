\chapter{Introduction}

\section{The Standard Model}
\label{SM}
For the last 50 years particle physicists have very successfully described the short distance interactions of elementary particles using the so called Standard Model.  
The Standard Model consists of two quantum field theories: the Electroweak Theory and \gls{qcd} Quantum Chromodynamics (or QCD), which describe the interactions of 12 fundamental spin $\frac{1}{2}$ fermions.  
While the Standard Model contains no explanation for gravity, dark matter, dark energy, etc. it does remain the most successful model available to describe reality on the smallest scales and at the highest energies. \\ 

\begin{table}
  \centering
  \begin{tabular}{ |c|c|c|c|c|c|}
  \hline
   fermion & mass [GeV] & spin & electric charge & colour charge & generation \\
  \hline
  \multicolumn{6}{|c|}{charged leptons} \\
  \hline
    e    & 5.11 x$10^{-4}$ & $\frac{1}{2}$ & -1 & no & 1 \\
  $\mu$  & 9.5 x$10^{-2}$  & $\frac{1}{2}$ & -1 & no & 2 \\
  $\tau$ & 1.78            & $\frac{1}{2}$ & -1 & no & 3 \\
  \hline
  \multicolumn{6}{|c|}{neutral leptons} \\
  \hline
  $\nu_e$      & 0 & $\frac{1}{2}$ & 0 & no & 1 \\
  $\nu_{\mu}$  & 0 & $\frac{1}{2}$ & 0 & no & 2 \\
  $\nu_{\tau}$ & 0 & $\frac{1}{2}$ & 0 & no & 2 \\
  \hline
  \multicolumn{6}{|c|}{up-type quarks} \\
  \hline
  u & 2.3 x$10^{-3}$ & $\frac{1}{2}$ & $+\frac{2}{3}$ & yes & 1 \\ 
  c & 1.28           & $\frac{1}{2}$ & $+\frac{2}{3}$ & yes & 2 \\
  t & 173.5          & $\frac{1}{2}$ & $+\frac{2}{3}$ & yes & 3 \\
  \hline
  \multicolumn{6}{|c|}{down-type quarks} \\
  \hline
  d & 4.8 x$10^{-3}$ & $\frac{1}{2}$ & $-\frac{1}{3}$ & yes & 1 \\
  s & 9.5 x$10^{-2}$ & $\frac{1}{2}$ & $-\frac{1}{3}$ & yes & 2 \\
  b & 4.18           & $\frac{1}{2}$ & $-\frac{1}{3}$ & yes & 3 \\
  \hline
  \end{tabular}
  \caption[Properties of known spin-$\frac{1}{2}$ bosons in the Standard Model.]
        {\small Properties of the known spin-$\frac{1}{2}$ fermions in the Standard Model~\cite{PDG}.  The quark masses have been estimated using the $\overline{\mathrm{MS}}$ renormalization scheme at a scale $\mu = 2$ GeV, and while the neutrino masses are non-zero, they are small enough that they are approximated as zero for the purposes of this thesis.}
\label{table:Fermions}
\end{table}


The fermions, listed in Table~\ref{table:Fermions}, are categorized into three generations of four particles, with each generation being a heavier copy of the previous generation.  
Each generation includes two quarks, particles which have colour charge and are therefore subject to the strong force described by QCD, and two leptons which have no colour charge.  
The six types (flavours) of quarks are organized into generational pairs, the up (u) and the down (d), the charm (c) and the strange (s), and the top (t) and the bottom (b).  
Similarly each generation of leptons is composed of one charged lepton, known as the electron (e), the muon ($\mathrm{\mu}$), and the tau ($\mathrm{\tau}$), and their neutral counterparts the neutrinos ($\nu_{\mathrm{e}}, \nu_{\mu}, \nu_{\tau}$).  
In additions to each of these matter particles there is a complimentary anti-particle, which have the opposite charge of their regular counter parts and are denoted using an over bar.  
It is worth noting that the entirety of the periodic table of elements, and therefore all regular matter, require only the first generation of particles.  
The Standard Model also includes spin 1 bosons which mediate the various forces included in the theory (see Table~\ref{table:Bosons}). \\

\begin{table}
  \centering
  \begin{tabular}{ |c|c|c|c|c|c|}
  \hline
  interaction     & boson             & mass [GeV] & spin & electric charge &  colour charge \\ \hline
  \multicolumn{6}{|c|}{force carrying bosons} \\ \hline
  electromagnetic & $\gamma$ (photon) & 0          & 1 & 0 & no \\ \hline
  \multirow{2}{*}{Weak} & $W^{\pm}$ & 80.39 & 1 & $\pm1$ & no  \\ 
                        & Z & 91.19 & 1 & 0 & no \\ \hline
  strong & g (gluon) & 0 & 1 & 0 & yes \\ \hline 
  \multicolumn{6}{|c|}{non-force carrying bosons} \\ \hline
  --- & Higgs & 125.09 & 0 & 0 & no \\ \hline
  \end{tabular}
  \caption[Properties of known bosons in the Standard Model.]
        {\small Properties of known bosons in the Standard Model.}
\label{table:Bosons}
\end{table}



The electroweak interaction has four force carrying bosons: the photon, the Z, and W$^{\pm}$.  
As the name suggests, while the electroweak force is one unified for at high energies at lower every day energies the electroweak symmetry is spontaneously broken into the electromagnetic force and the weak force.  
In the Standard Model this breaking is explained by introducing a new scalar field with a non-zero ground amplitude in the ground state, known as the Higgs field.  
This breaking into two separate forces also allows the W and Z bosons to acquire their very large observed mass through the Brout-Englert-Higgs (BEH) mechanism.  
This Higgs field also allows the fermions in the Standard Model to obtain mass as well.  
A measurable consequence of this theory is the presence of a massive spin-0 boson called the Higgs Boson, which was discovered in 2012 at the Large Hadron Collider (LHC).  \\

As previously mentioned, QCD describes the strong force, which affects particles carrying colour charge much like the electric force affects particles carrying an electric charge.  
Unlike the electromagnetism which only has one charge, the electric charge, QCD has three colour charges known as red, green, and blue.  
Another notable difference between the two is that while the force mediator of the electric force (the photon) contains no electric charge itself, this is not the case for the force carriers of the strong interaction (gluons).  
One important consequence of this difference is that the strength of the strong force does not weaken with increasing distance between the particles in question.  
This difference means that quarks cannot be found as individual particles but come in bound `colour neutral' states, a phenomenon known as confinement.  
While these bound states are colour neutral a much weaker residual force still exists, and while it decreases quite rapidly with distance it is this residual force which is responsible for holding protons and neutrons together in the nuclei of atoms.  \\

These bound states contain an infinite number of quarks/anti-quark pairs constantly being created and annihilated (called sea quarks) and gluons, but the identity and properties of a given bound state are determined by their so called valence quarks.  
Bound states can have either three quarks (or anti-quarks) which are known as the baryons, or a quark/anti-quark pair, the mesons.  
As hinted at above, familiar examples of these bound states include protons (uud) and neutrons (udd).  
As a result of these complicated bound states the masses of the quarks cannot be measured directly, but must be approximated by measuring the bound states and making certain theoretical assumptions~\footnote{The t quark is the exception, as it decays fast enough that no bound states are formed.}.
 

\section{Experimental Particle Physics}
\label{Sec:Experi}
The Standard Model and all of it's predictions can be studied in a variety of settings.  
One category of experiment involves accelerating long lived stable particles and bringing them to collision inside of a detector, allow for the study of the shorter lived particles of the Standard Model.  
The particles being accelerated can be both elementary (electrons for example) or composite (like protons), with difference the different choices effecting the types of collisions which will be observed.  
A relevant example of such an accelerator/detector combination is the Large Hadron Collider (LHC), a proton-proton machine producing collisions inside the ATLAS (A Torroidal LHC ApparatuS) detector, both of which will be covered in Chapter~\ref{Experiment}.  \\

\section{Units and Conventions}
 






