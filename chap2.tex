\chapter{Experimental setup}
\label{Experiment}
The \gls{LHC} was able to keep its first protons circulating within its 27 km of underground tunnel on the 10th of September, 2008.  
On the 23rd of November the following year the first collisions were recorded~\cite{FirstCollisions} signalling the beginning of the LHC's era at the forefront of the high energy frontier.  
In the years that have passed since the LHC has been consistently pushing to more frequent collisions at higher energies, allowing the experiments along its ring to explore and extend the boundaries of human knowledge.  

This chapter will describe the technology that allows these studies to be performed.  
First the LHC acceleration complex will be provided.  
This will be followed by a description of the ATLAS detector which will be broken into the various subdetectors which make up ATLAS and the methods they use to accomplish their given tasks.   

\section{The Large Hadron Collider}
\label{Sec:LHC}

The parts of a particle accelerator can be broken down into two functional categories: rf cavities which use electric fields to accelerate particles and magnets which are used to confine the accelerated particles.  
Circular accelerators come in two distinct flavours: cyclotrons which contain the beam using a fix strength magnetic field but allow for the radius of the circle to increase, and synchrotrons which maintain a fixed radius by increasing the magnetic field strength to compensate for the increasing energy of the accelerating particles.  
The LHC fits into the second of these categories.  

Designing to single synchrotron accelerator to accelerate bunches of protons to the energies attained at the LHC would be a difficult and expensive task.
One requirement would be the ability to finely control the field in the superconducting electromagnets used to contain the particle beam in the ring over several orders of magnitude.  
A much more cost effective approach would involve using existing machines to accomplish at least part of this task, and that's exactly the approach used at the LHC.  

The oldest component of the modern day LHC acceleration facility is the Proton Synchrotron, or PS, which was built in the 1950's which was designed to accelerate up to 10$^{10}$ protons per pulse to 26 GeV.  
The low energy at which protons were being injected into the PS was limiting the number of protons which could be injected into the ring, so in 1972 the PS booster was added.  
The PS Booster consists of four superimposed synchrotron rings which accelerated protons to 1.4 GeV before they entered the PS, allowing the PS to contain as many as 10$^{13}$ protons.  
In 1978 the linear accelerator which had been injecting protons into the PS Booster, known as the linac, was replaced by linac2, a machine which accelerates protons originating in a bottle of hydrogen gas to 50 MeV in longer pulses than its predecessor, allowing the PS Booster to be used to its full capability~\cite{PSHistory}~\cite{PSHistory2}.  

After the PS the protons are injected into the Super Proton Synchrotron (SPS), a machine which in 1976 when it was turned on had the ability to accelerate protons to 400 GeV, the second highest energy attainable by accelerator at that time.  
The SPS may be most well known as the machine that allowed for the discovery of both the W and Z bosons by the UA1 and UA2 collaborations.  
It may be worth noting that it takes several fills of the PS to fill the SPS, and it will again take several fills of the SPS to fully fill the LHC.  
The SPS is now the final pre-injector into the LHC, accelerating the protons up to 450 GeV before they enter the LHC itself.  

The LHC is a machines that was designed to use 16 RF cavities to accelerate 2808 bunches, each with 1.15x$10^{11}$ protons, to 7 TeV per proton.  
These bunches travel the 27 km around the ring of the LHC being held in place by 1232 8.33 Tesla superconducting dipole magnets which bend the beam, along with a number of additional magnets which help keep the beam focused~\cite{LHCTDR}.  
The LHC then crosses these counter rotating beams at four points along the beam line (the four experiments listed in~\ref{Sec:Experi}) every 25 ns, where up to 40 interactions are recorded per crossing.  

One measure of the collider's performance is it's luminosity, a description of the number of particles per unit area per unit time, which directly relates to the number of collisions per unit time and therefore the probability of obtaining a given final state per unit time.  
The LHC's design luminosity is approximately 20 times the maximum luminosity of its predecessor, the Tevatron ($10^{34}$ cm$^{-2}$s$^{-1}$ vs. 4 x $10^{32}$ cm$^{-2}$s$^{-1}$), and early in the 2016 data taking period the LHC has already started recording personal highs (8.8 x $10^{33}$ cm$^{-2}$s$^{-1}$

\section{The ATLAS Experiment}
\label{Sec:ATLAS}

\subsection{The ATLAS Coordinate System}

Before laying out the design and components of the ATLAS detector we should first establish the coordinate which will be used.  
The origin of the coordinate system that is used in ATLAS is at the geometric centre of the ATLAS detector.  
The z-axis runs parallel to the beam pipe running counterclockwise along the LHC ring when viewed from above, with the x-axis pointing toward the middle of the ring and the y-axis pointing up.  
A more common way to describe the coordinates in the x-y plane is to use the azimuthal angle $\phi$, where $\phi$=0 is defined to be along the positive x-axis and to increase in the direction of the positive y-axis.

A similar coordinate $\theta$ can be defined with $\theta$=0 being in the x-y plane with $\theta$ increasing in the direction of the positive z-axis, although it is often more convenient to speak in terms of rapidity($y$) and pseudorapidity($\eta$).  
This is because of how the $\theta$ coordinate transforms as one transitions from the detector reference frame to the centre of mass reference frame for the collisions the detector is measuring.  
Rapidity, given by $y=\frac{1}{2}\mathrm{ln}\left[\frac{E+p_{z}}{E-p_z}\right]$, is much more helpful in this regard as the difference in rapidity remain constant when moving between reference frames.  
Pseudorapidity, defined as $\eta=-\mathrm{ln}\left[\mathrm{tan}\left(\frac{\theta}{2}\right)\right]$, acts as a compromise between these two choices.  
Pseudorapidity is equal to the rapidity in the case of a massless particle so it maintains some connection to the underlying physics of a given event while not introducing a mass dependence in the connection between the coordinate system and the detector.  


\subsection{ATLAS Detector: Overview}

ATLAS is a multipurpose detector, meaning it must be able to simultaneously measure the large number and variety of particles produced in each collision provided by the LHC at a high enough rate to take advantage of all of the data being provided.  
The ATLAS detector is made up of individual layers, with each layer being designed to optimize the measurement of a different types of signals.  
Going from the interaction point outwards these layers are known as the inner detector, the electromagnetic (EM) calorimeter, the hadronic calorimeter, and the muon spectrometer.   

\subsection{ATLAS Hardware: Inner Detector}

The inner detector is made up of three subdetectors: the pixel detector, the semiconductor tracker (SCT), and the transition radiation tracker (TRT).  
These three subdetectors are all immersed in a 2 Tesla magnetic field that is being supplied by a solenoidal magnet which encompasses the entire inner detector.  
This magnetic field bends the trajectories of charged particles by an amount that is proportional to their momentum.  
The main purpose of the inner detector is to make non destructive measurements of this bending, allowing the momentum of charged particles to be measured.  
It is also possible to track the trajectories of multiple particles back to a single origin, a so called vertex.  
Vertices along the beam axis may indicate an individual proton-proton collision event, while vertices off of the beam axis may indicate the location where some heavy particle produced in the original collision further decays into lighter secondaries.  
 
The way these trajectories are measured is by having a large number of small detectors with well known positions surrounding the interaction point.  
When a particle passes through these small detectors a signal is measured (a hit).  
A reconstruction algorithm is then run over all of these hits to recreate all the paths the particles have travelled (known as tracks).  
It's this track reconstruction that leads the inner detector to more colloquially be known as the tracked.  

The first two subdetectors of the tracker are both semiconductor detectors, a type of detector that measures the electron-hole pairs that are produced as a charged particle passes through a silicon sensors that are segmented into either squares (pixels) or strips.  
Both are made up of 4 concentric cylinders in the central region, with circular endcaps further extending the $\eta$ acceptance of the detectors.  
For the 2011 and 2012 data taking periods the pixel detector consisted of three layers situated 50.5, 88.5 and 122.5 mm from the centre of the beam pipe.  
The layers are made up of 22, 38 and 52 staves, with each stave containing 6 x $10^5$ pixels.  
This detector was capped at both ends by 3 layers of endcap, with each endcap having a further 4.4 x $10^6$ pixels.  
The resolution of a tracking detector is parameterized by A $\oplus$ B/p$_{\mathrm{T}}$, where A is the intrinsic resolution of the detector and B describes the effect of multiple scatterings on the resolution.  
This setup allowed for an intrinsic resolution of 10 $\mu$m in 10 $\mu$m in the transverse impact parameter d$_{0}$ and 115 $\mu$m in the longitudinal impact parameter (z$_{0}$ sin$\theta$)~\cite{ID3}.  

During the first long shutdown of the LHC a fourth additional layer was added closer to the beam, known as the Insertable B-Layer (IBL).  
This layer was designed to both withstand high levels of radiation while enabling the ATLAS tracking performance to avoid degradation up to the potentially very high luminosities the LHC will be delivering by the year 2020 (2-3 x $10^{34}$ cm$^{-2}$s$^{-1}$)~\cite{IBL1}.  
This new layer has 14 staves which are arranged in an overlapping circular pattern 33 mm away from the centre of the beam pipe, adding a further 6 x $10^{6}$ individual readout pixels to the pixel detector.  
The IBL improves the intrinsic resolution of the pixel detector by a factor of 1.7 in longitudinal impact parameter and by 1.2 in transverse impact parameter.  
It also reduces the p$_{\mathrm{T}}$ dependence of the resolution by a factor of 1.8 in both directions~\cite{IBL2}.  
This improved resolution benefits many measurements, for example without pileup it increases the light jet rejection rate of a 60\% efficient b-tagger by nearly a factor of 2.  

As mentioned above the SCT is also a semiconductor detector, consisting of 4088 modules tiling 4 cylinders and two endcaps, with each endcap consisting of 9 layers~\cite{JOIATLAS}.  
In the barrel these modules consist of four sensors, two on the top and two on the bottom with a stero angle of 40 mrad to increase the resolution of the modules.  
The endcap modules use the same strategy, having two sensors glued back to back once again with a stero angle of 40 mrad.  
The SCT has a nominal resolution of 17 $\mu$m in R-$\phi$ and 580 $\mu$m in $z$.  

 



\subsection{ATLAS Hardware: Calorimeter}

\subsubsection{Electromagnetic Calorimeter}
\label{EMCalo} 

\subsubsection{Hadronic Calorimeter}
\label{Had}

\subsection{ATLAS Hardware: Muon Spectrometer}

\subsection{Triggers}
\label{Trig}
