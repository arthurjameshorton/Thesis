\chapter{Experimental setup}
\label{Experiment}
The \gls{LHC} was able to keep its first protons circulating within its 27 km of underground tunnel on the 10th of September, 2008.  
On the 23rd of November the following year the first collisions were recorded~\cite{FirstCollisions} signalling the beginning of the LHC's era at the forefront of the high energy frontier.  
In the years that have passed since the LHC has been consistently pushing to more frequent collisions at higher energies, allowing the experiments along its ring to explore and extend the boundaries of human knowledge.  \\

This chapter will describe the technology that allows these studies to be performed.  
First the LHC acceleration complex will be provided.  
This will be followed by a description of the ATLAS detector which will be broken into the various subdetectors which make up ATLAS and the methods they use to accomplish their given tasks.   

\section{The Large Hadron Collider}
\label{Sec:LHC}

The parts of a particle accelerator can be broken down into two functional categories: rf cavities which use electric fields to accelerate particles and magnets which are used to confine the accelerated particles.  
Circular accelerators come in two distinct flavours: cyclotrons which contain the beam using a fix strength magnetic field but allow for the radius of the circle to increase, and synchrotrons which maintain a fixed radius by increasing the magnetic field strength to compensate for the increasing energy of the accelerating particles.  
The LHC fits into the second of these categories.  \\

Designing to single synchrotron accelerator to accelerate bunches of protons to the energies attained at the LHC would be a difficult and expensive task.
One requirement would be the ability to finely control the field in the superconducting electromagnets used to contain the particle beam in the ring over several orders of magnitude.  
A much more cost effective approach would involve using existing machines to accomplish at least part of this task, and that's exactly the approach used at the LHC.  \\

The oldest component of the modern day LHC acceleration facility is the Proton Synchrotron, or PS, which was built in the 1950's which was designed to accelerate up to 10$^{10}$ protons per pulse to 25 GeV.  
The low energy at which protons were being injected into the PS was limiting the number of protons which could be injected into the ring, so in 1972 the PS booster was added.  
The PS Booster consists of four superimposed synchrotron rings which accelerated protons to 1.4 GeV before they entered the PS, allowing the PS to contain as many as 10$^{13}$ protons.  
In 1978 the linear accelerator which had been injecting protons into the PS Booster, known as the linac, was replaced by linac2, a machine which accelerates protons originating in a bottle of hydrogen gas to 50 MeV in longer pulses than its predecessor, allowing the PS Booster to be used to its full capability~\cite{PSHistory}~\cite{PSHistory2}.  \\



 
 

\section{The ATLAS Experiment}
\label{Sec:ATLAS}

\subsection{The ATLAS Coordinate System}

\subsection{ATLAS Detector: Overview}

\subsection{ATLAS Hardware: Inner Detector}

\subsection{ATLAS Hardware: Calorimeter}

\subsubsection{Electromagnetic Calorimeter}
\label{EMCalo} 

\subsubsection{Hadronic Calorimeter}
\label{Had}

\subsection{ATLAS Hardware: Muon Spectrometer}

\subsection{Triggers}
\label{Trig}
