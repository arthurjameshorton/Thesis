\chapter{Additional Jet Studies}

Even with a jet calibration firmly in hand additional studies on jet phenomenology can go a long way towards advancing our knowledge of the jet response and potentially moving towards better calibrations in the future.  
In this chapter the question of how the energy from the hadronic recoil in the 2$\rightarrow$2 systems used in \textit {in-situ} calibration schemes is distributed around the jet axis is explored.  
The flow of energy across the boundary of the jet definition is also studied (see the showering correction in Sec.~\ref{sec:ShoweringIntro}).  
Finally some key assumptions used in the derivation of the MPF are examined, including the relationship between the response of the recoil and the response of the jet, as well as the effect of the underlying event on the derivation of the MPF.  

\section{Distribution of Hadronic Recoil Energy}
\label{Sec:EnergyDensity}
\begin{figure}[!ht]
 \begin{center}
  \scalebox{0.55}{
   \includegraphics{plots/Chap6/Rings/EM/ResponseVsr_17-20_log.eps}
  }
 \end{center}
 \caption[Recoil energy distribution in Z+jet, 17-20 GeV]
 {\small Average distribution of energy surrounding the leading jet in Z+jet events with $p_{\mathrm T}^{Z}$ between 17 and 20 GeV. }
 \label{Fig:EMShape17-20}
\end{figure}

As mentioned in Sec.~\ref{sec:ShoweringIntro} the majority of the energy from the hadronic recoil resides within a narrow energy dense core (the jet).  
In this section this statement is explored by measuring how the energy from the recoil is distributed around the calorimeter.  
This is done by measuring the average energy density in an annulus with an inner radius of $r-\Delta r$ and an inner radius of $r+\Delta r$, which is defined as
\begin{equation}
  \rho\left(r\right)  = \left( \frac{ E\left(r-\Delta r, r+\Delta r\right)}{\pi \left[\left(r +\Delta r\right)^2-\left(r -\Delta r\right)^2\right] E^{\mathrm{jet}}}\right), 
\end{equation}
where $E\left(r-\Delta r, r+\Delta r\right)$ is the sum of the energy of all clusters with a distance $R$ from the jet between $r-\Delta r$ and  $r+\Delta r$.  
As this thesis aims to measure this energy density profile in the context of an MPF based calibration results are obtained using the same selection used in the main analysis (see Sec.~\ref{Sec:SelectionCriteria}), and is also binned in $p_{\mathrm T}^{\mathrm{Ref}}$.  
The energy distribution using EM scale clusters at three different energies with Z+jet events are shown in Figs.~\ref{Fig:EMShape17-20},~\ref{Fig:EMShape60-80}, and~\ref{Fig:EMShape210-260}.  
One feature that should be understood is a dip in energy density at the boundary of the jet (in this case at $\Delta R = 0.4$).  
This dip is not a feature of the recoil or the calorimeter, but of the jet reconstruction algorithm, which creates an object which is centered on the most energetic clusters it contains.  
This dip can be explained by imagining a $\Delta R = 0.35$ jet growing into a $\Delta R = 0.4$ jet.  
If this jet growth only adds clusters with energies which are small with respect to the energy of the growing proto-jet the centre of the jet (the four-momentum sum of its constituents) does not move very much.  
If there is a cluster with a relatively high energy compared to the energy of the proto-jet the centre of the jet will move closer to this new addition.  
This process ensures that high energy clusters within a jet cannot be near the jet boundary.  
This process is especially evident for lower energy jets where even very small energy clusters represent a relatively large fraction of the total jet energy.  


\begin{figure}[!ht]
 \begin{center}
  \scalebox{0.55}{
   \includegraphics{plots/Chap6/Rings/EM/ResponseVsr_60-80_log.eps}
  }
 \end{center}
 \caption[Recoil energy distribution in Z+jet, 60-80 GeV]
 {\small Average distribution of energy surrounding the leading jet in Z+jet events with $p_{\mathrm T}^{Z}$ between 60 and 80 GeV. }
 \label{Fig:EMShape60-80}
\end{figure}

At larger distances from the jet's core the energy density is relatively flat, which corresponds to the average pileup energy density.  
There is also a slight increase in the energy density with growing distance from the leading jet.  
This is caused by the presence of the subleading jet, which tends to be produces in the same hemisphere of the calorimeter as the reference object.  


\begin{figure}[!ht]
 \begin{center}
  \scalebox{0.55}{
   \includegraphics{plots/Chap6/Rings/EM/ResponseVsr_210-260_log.eps}
  }
 \end{center}
 \caption[Recoil energy distribution in Z+jet, 210-260 GeV]
 {\small Average distribution of energy surrounding the leading jet in Z+jet events with $p_{\mathrm T}^{Z}$ between 210 and 260 GeV. }
 \label{Fig:EMShape210-260}
\end{figure}

\section{Expanded MPF Derivation}

In Sec.~\ref{METProj} a derivation of the MPF was presented which considered only the hard scattering event, ignoring any potential effects of the underlying event.  
This potential deficiency in the derivation was justified by arguing that the underlying event is symmetric in $\phi$ when averaged over many events and therefore does not contribute to the measured MPF response.  
This assumption is tested in this section.  

The definition of the MPF in Eq.~\ref{EQ:MPFSimple}, is modified by a term $\Delta^{\mathrm{OA}}$ that accounts for the effect of other activity in the event on the MPF, 
\begin{equation}
  \label{Eq:MPFWithDelta}
  R_{\mathrm{recoil}}=1+\frac{\vec{E}_{\mathrm T}^{\mathrm{miss, recoil}}\cdot\hat{p}_{\mathrm T}^{\mathrm{ref}}}{E_{\mathrm T}^{\mathrm{ref}}} + \Delta^{\mathrm{OA}}, 
\end{equation}
\noindent
where it is explicitly shown that the original derivation assumed that the MET originated completely from the mismeasurement of the energy of the recoil.  
In an effort to obtain an expression in terms of measurable quantities $\vec{E}_{\mathrm T}^{\mathrm{miss, recoil}}$ is now expressed as 

\begin{equation}
  \vec{E}_{\mathrm T}^{\mathrm{miss, recoil}} = \vec{E}_{\mathrm T}^{\mathrm{recoil}} - \vec{E}_{\mathrm T}^{\mathrm{meas, recoil}}, 
\end{equation}
\noindent 
which is to say the amount of energy that isn't measured is equal to the total energy minus the energy that has been measured.  
Using the response $R_{\mathrm{recoil}}$ of the calorimeter for the recoil, 
\begin{equation}
  R_{\mathrm{recoil}}\vec{E}_{\mathrm T}^{\mathrm{recoil}} = \vec{E}_{\mathrm T}^{\mathrm{meas, recoil}}, 
\end{equation}
\noindent
which leads to 
\begin{equation}
  \vec{E}_{\mathrm T}^{\mathrm{miss, recoil}} = \vec{E}_{\mathrm T}^{\mathrm{meas, recoil}}\left(\frac{1-R_{\mathrm{recoil}}}{R_{\mathrm{recoil}}}\right).  
\end{equation}
\noindent
Using this definition of $\vec{E}_{\mathrm T}^{\mathrm{miss, recoil}}$ in Eq.~\ref{Eq:MPFWithDelta} and rearranging to solve for $\Delta^{\mathrm{OA}}$, the following expression is obtained for the contribution of the other activity to the MPF 
\begin{equation}
  \Delta^{\mathrm{OA}}=\left(R_{\mathrm{recoil}}-1\right)\left(1+\frac{\hat n_{\mathrm{ref}}\cdot \vec{E}_{\mathrm T}^{\mathrm{meas, recoil}}}{R_{\mathrm{recoil}}{E}_{\mathrm T}^{\mathrm{ref}}}\right).  
  \label{Eq:OA}
\end{equation} 
To use this expression both the response of the recoil and the measured recoil energy must be known, neither of which is true.  
To first order the effect of the other activity can be assumed to be small (as the MPF has been successfully used in jet calibration schemes in the past).  
This allows the use of the MPF itself as a measure of the response of the recoil.  
Obtaining a measurement of the full energy of the hadronic recoil at the detector level with the underlying event removed (and also without including pileup) is also a difficult issue.  
As the MPF technique already intimately connects the energy of the recoil to the energy of the jet this strategy will be used again.  
A range of jet cone sizes will be used ranging from a jet size parameter of $R$=0.1, which contains only the highest energy most central components of a jet, to $R$=1.0 which contains the vast majority of the hard scattering along with a significant amount of underlying event.   
As these are non-standard jet collections within the ATLAS collaboration the residual pileup correction is not available for all sizes considered, however it is still possible to remove the majority of the pileup from each jet using the jet area subtraction as described in Sec.~\ref{ATLASJES}.  

It is worth noting a few limitations of this technique before moving further.  
Just like in the derivation of the MPF substituting the energy of the leading jet for the energy of the recoil is a sensible substitution only when the leading jet does make up the majority of the recoil (when the balance is not spoils by ISR/FSR).  
This is ensured once again by applying both a $\Delta\phi$ and subleading jet (J2) requirement.  
In the case of this analysis $\Delta\phi\left(\mathrm{leading jet, ref}\right)$>2.9 and the $p_{\mathrm T}$ of the subleading jet without calibration must be less than 5\% of the $p_{\mathrm T}$ of the reference object ($p_{\mathrm T}^{J2}<0.05$ $p_{\mathrm T}^{\mathrm{Ref}}$).  
At low energies pileup becomes a larger issue, first by pileup jets being misidentified as the subleading jet which changes how strict the previous requirement is applied, and then by pileup jets being misidentified as the leading jet causing the event to fail the $\Delta\phi$ requirement.  
An example of how these effects change the measured response distribution in a $p_{\mathrm T}$ based response measurement can be seen in Fig.~\ref{Fig:BalDistExample}.  

\begin{figure}[!ht]
 \begin{center}
  \scalebox{0.55}{
   \includegraphics[width=0.9\linewidth]{plots/Chap6/OtherActivity/Dists/OADist_Cone7_Bin3.eps}
%BalanceOtherActivity7_bin_3_30.00_log.eps}
  }
 \end{center}
 \caption[Example $\Delta^{\mathrm{OA}}$ distribution]
 {\small Distribution of $\Delta^{\mathrm{OA}}$ in 2016 data at EM scale using Z+jet events, with 30 GeV $< p_{\mathrm T}^Z <$ 35 GeV.  Made using anti-$k_{\mathrm t}$ $r$=0.7 jets as an estimate for the hadronic recoil energy.  }
 \label{Fig:BalDistExample}
\end{figure}


\begin{figure}[!ht]
  \centering
  \begin{subfigure}{.5\textwidth}
    \centering
    \scalebox{0.4}{
      \includegraphics{plots/Chap6/OtherActivity/Z/OtherActivity1VsPt_Z_EM_Data.eps}
    }
    \caption{Anti-$k_\mathrm{t}$ R=0.1}
  \end{subfigure}%
  \begin{subfigure}{.5\textwidth}  \centering
   \scalebox{0.4}{
      \includegraphics{plots/Chap6/OtherActivity/Z/OtherActivity4VsPt_Z_EM_Data.eps}
    }
    \caption{Anti-$k_\mathrm{t}$ R=0.4}
  \end{subfigure}
  \caption[$\Delta^{\mathrm{OA}}$ using Anti-$k_\mathrm{t}$ R=0.1/0.4 jets]
{\small The effect of the other activity in the event on the measured recoil response, as determined using Eq.~\ref{Eq:OA}.  The leading Anti-$k_\mathrm{t}$ R=0.1/0.4 jet reconstructed using EM scale clusters is used as an estimate for the energy of the full recoil.  Using data collected in 2016.  }
  \label{Fig:OA_1-4}
\end{figure}


\begin{figure}[!ht]
  \centering
  \begin{subfigure}{.5\textwidth}
    \centering
    \scalebox{0.4}{
      \includegraphics{plots/Chap6/OtherActivity/Z/OtherActivity7VsPt_Z_EM_Data.eps}
    }
    \caption{Anti-$k_\mathrm{t}$ R=0.7}
  \end{subfigure}%
  \begin{subfigure}{.5\textwidth}  \centering
    \scalebox{0.4}{
      \includegraphics{plots/Chap6/OtherActivity/Z/OtherActivity10VsPt_Z_EM_Data.eps}
    }
    \caption{Anti-$k_\mathrm{t}$ R=1.0}
  \end{subfigure}
  \caption[$\Delta^{\mathrm{OA}}$ using Anti-$k_\mathrm{t}$ R=0.7/1.0 jets]
{\small The effect of the other activity in the event on the measured recoil response, as determined using Eq.~\ref{Eq:OA}.  The leading Anti-$k_\mathrm{t}$ R=0.7/1.0 jet reconstructed using EM scale clusters is used as an estimate for the energy of the full recoil.  Using data collected in 2016.  }
  \label{Fig:OA_7-10}
\end{figure}

This measure of the effect of the other activity on the MPF is shown as a function of $p_{\mathrm T}^{\mathrm Ref}$ using various jet sizes as a substitute for the energy of the recoil in Figs.~\ref{Fig:OA_1-4}~and~\ref{Fig:OA_7-10}.  
As seen in Sec.~\ref{Sec:EnergyDensity} the energy density of jets in the energy range considered remains large compared to that of pileup/underlying activity out to at least a distance of $\Delta\mathrm{r}=$ 0.4 away from the core.  
For this reason the use of jets reconstructed using size parameters lower than this value as a stand in for the energy of the recoil is known to be a bad approximation, and therefore this plots are included only for illustrative purposes.  
As shown in Fig.~\ref{Fig:OA_ConeSize} $\Delta^{\mathrm{OA}}$ can be quite sensitive to the choice of cone size at small values but it tends to show that the MPF is not effected by the other activity in the even when using a suitably large radius jet which includes the full recoil.  


\begin{figure}[!ht]
  \centering
  \begin{subfigure}{.5\textwidth}
    \centering
    \scalebox{0.4}{
      \includegraphics{plots/Chap6/OtherActivity/Z/OAVsCone_60-80.eps}
    }
    \caption{60 GeV $< p_{\mathrm{T}}^{\mathrm{ref}} < $ 80 GeV}
  \end{subfigure}%
  \begin{subfigure}{.5\textwidth}  \centering
    \scalebox{0.4}{
      \includegraphics{plots/Chap6/OtherActivity/Z/OAVsCone_350-800.eps}
    }
    \caption{350 GeV $< p_{\mathrm{T}}^{\mathrm{ref}} < $ 800 GeV}
  \end{subfigure}
  \caption[$\Delta^{\mathrm{OA}}$ using various cone sizes]
{\small The effect of the other activity in the event on the measured recoil response, as determined using Eq.~\ref{Eq:OA}.  The dependence of this quantity on the size of the jet used as a stand in for the total recoil is shown for two $p_{\mathrm{T}}^{\mathrm {ref}}$ bins.  Results shown have been measured using EM scale jets from the 2016 ATLAS dataset.  }
  \label{Fig:OA_ConeSize}
\end{figure}











