\chapter{Conclusion}

As the LHC collides protons, every collision results in some hadronic activity in the final state.  
This means that a precise knowledge of the jet energy scale is an important factor to the success of the ATLAS physics program.  
This thesis measures the jet energy scale in ATLAS using the missing transverse energy projection fraction method (MPF) applied to both Z+jet and $\gamma$+jet events.  
In both channels the response of both EM and LC scale jets has been measured.  
In moving from the $\sim$ 4 fb$^{-1}$ collected in 2015 to the $\sim$ 35 fb$^{-1}$ collected in 2016 the uncertainties with the Z+jet channel for both scales was reduced from $\sim$ 1.2\% to $\sim$ 0.8\% between 30 GeV and 500 GeV.   
The uncertainty on the $\gamma$+jet channel was also reduced, remaining slightly below 1\%,  while extending the range of validity of the calibration from 700 GeV to 1.2 TeV.  
Moving forward these studies will be expanded to include a third scale of jet inputs known as particle flow objects which combine both calorimeter and tracking information.  
In addition to producing the jet calibration required for the ATLAS physics program, several studies were done to understand better and validate the MPF technique.  
Many of the assumptions previously made were tested.

The energy density of the hadronic recoil as a function of distance from the reconstructed jet axis has been studied.  
The excess in energy density caused by the hadronic recoil above the underlying event is found to be mostly contained within a distance of R=0.7, with the majority of this excess being comfortably contained within a cone of R=0.5 for jets with energy above 60 GeV.  
Studies of the effect of pileup and other activity on the MPF's ability to measure the response of the calorimeter to the hadronic recoil are also presented.  
These studies use a number of different size parameters when reconstructing jets, using large jet sizes as a proxy for the full hadronic recoil.  
These studies show that the assumption that this additional energy in the event does not affect the measurement is in fact valid.  


Studies on the effect of the flow of energy across the jet reconstruction boundary have also been performed by way of the so-called showering correction.  
The showering correction shows that the amount of energy leaving the jet is larger than the amount entering from other activity, and that on average a low energy ($\sim$ 30 GeV) anti-$k_\mathrm{t}$ R=0.4 jet would have a response 6\% higher if this effect was removed.  
The size of this effect becomes smaller both with increasing jet energy and increasing jet size.  
The showering correction was shown to vary by less than 0.5\% with the choice of physics list, which, among other things, models the development of the calorimeter shower.

Finally the importance of the low energy/low response particles near the fringe of the hadronic recoil to the response of the total recoil is studied by way of the topology correction.  
These low energy particles are found to bring the response of the recoil down approximately 5\% compared to the response of the core, for the case of an anti-$k_\mathrm{t}$ R=0.4 jet.  
This difference once again decreases with increasing jet size.  
This quantity is unaffected by the choice of physics list above 30-40 GeV. 
However, the choice of the physics list changes the impact of the jet reconstruction threshold at low pt.  

The dependence on the choice of physics list for these two quantities can be added as a uncertainty on the MPF method for in-situ calibration.  
The uncertainty on these effects have previously been covered by adding the uncertainty from the so-called out-of-cone (OOC) correction for the $p_{\mathrm{T}}$ balance method, an unrelated quantity which was believed to be a conservative estimate on the uncertainty.  
Using the dependence of the measured total correction on the choice of physics list as a measure of the uncertainty on this quantity in place of the OOC uncertainty leads to a reduction from 3\% to 0.5\% uncertainty in the lowest energy bin for the Z+jet analysis, with the uncertainty remaining approximately unchanged above 35 GeV.  
In the $\gamma$+jet channel after the rebinning described in Sec.~\ref{Sec:Systs} there is no depence of this total correction on the choice of physics list.  
This is a recution from the OOC uncertainty which in 2016 was approximately 1\% below 50 GeV and 0.5\% in the 50-100 GeV range.  
These more accurate uncertainties reduce the overall uncertainty on the JES for jets below 100 GeV.  
Since the JES uncertainty is the largest contribution to the total uncertainty in many ATLAS analyses, the work described in this thesis will noticeably improve ATLAS results at low energy.

 



