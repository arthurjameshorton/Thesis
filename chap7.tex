\chapter{Conclusion}
Jets are the most abundantly produced physics objects created in high-energy proton-proton collisions.  
As a result, a precise knowledge of the jet energy scale is essential to the success of ATLAS.  
In this thesis the jet energy scale is measured using the missing transverse energy projection fraction method applied to Z+jet events.  
Cases where the Z boson decays into an electron-positron pair, as well as into a pair of muons are considered.  
In both cases the jet response has been determined using jets which have been reconstructed at EM and LCW scale.  
In all cases, \emph{in-situ} results are found to agree with simulation within uncertainty.  \\

This study will be extended to include the full 20 fb$^{-1}$ of 8 TeV data taken in 2012.  
Along with a reduction on the statistical uncertainty of the response measurement, the varying pile-up conditions on the additional data will allow for better exploration of the effects of pile-up on the measurement itself.  
The full data set may also be large enough to allow for accurate measurements of the jet response in the forward regions of the calorimeter, which would be combined with dijet and gamma jet measurements to decrease the uncertainty on the jet energy scale at larger pseudorapidities.  \\

Jets initiated by gluons tend to contain a higher number of lower energy particles compared to jets initiated by light quarks.  
This results in gluon-initiated jets having a lower response.  
By comparing response measurements performed in channels with different fractions of events being initiated by gluons, a parton-dependent jet response can be obtained.  
In this thesis a Monte Carlo based study has been performed using $\gamma$+jet and Z+jet samples, as well as a preliminary data-based measurement.  
While results consistent with expectation have been achieved, large statistical errors are present in the gluon jet response measurement due to the low fraction of gluon jets in the $\gamma$+jet sample.  
Improved parton-dependent responses may be obtained by including additional jet response measurements derived in channels with differing quark/gluon composition (dijets for example).   




