\chapter{Conclusion}

As the LHC collides protons every collision results in some hadronic activity in the final state.  
This means that a precise knowledge of the jet energy scale is an important factor to the success of the ATLAS physics program.  
This thesis measures the jet energy scale in ATLAS using the missing transverse energy projection fraction method applied to both Z+jet and $\gamma$+jet events.  
In both channels the response of both EM and LC scale jets have been measured.  
In moving from the $\sim$ 4 fb$^{-1}$ collected in 2015 to the $\sim$ 35 fb$^{-1}$ fb collected in 2016 the uncertainties with the Z+jet channel for both scales was reduced from $\sim$ 1.2\% to $\sim$ 0.8\% between 30 GeV and 500 GeV.   
The same move slightly reduced the uncertainty on the $\gamma$+jet channel (which remained slightly below 1\%) while extending its range from 700 GeV to 1.2 TeV.  
Moving forward these studies will be expanded to include a third scale of jet inputs known as particle flow objects which combine both calorimeter and tracking information.  

The energy density of the hadronic recoil as a function of distance from the reconstructed jet axis has been studied, with the increase in energy density due to the recoil becoming negligible beyond a distance of R=0.7, leaving only the energy of both pileup and the hadronic recoil.  
Studies into the effect of pileup and other activity on the MPF's ability to measure the response of the hadronic recoil are also presented.  
These studies use a number of different size parameters when reconstructing jets, using large jet sizes as a proxy for the full hadronic recoil.  
These studies show that the assumption that this additional energy in the event does not effect the measurement is in fact a valid assumption.  


Studies on the effect of the flow of energy across the jet reconstruction boundary have also been performed by way of the so-called showering correction.  
The showering correction shows that the amount of energy leaving the jet is larger than the amount entering the jet, and that on average a low energy jet ($\sim$ 30 GeV) anti-$k_\mathrm{t}$ R=0.4 would have a response 6\% higher if this effect was removed, with this difference decreasing with increasing jet energy.  
The size of this effect also decreases with increasing jet size.  
The size of this effect was found to have a sub 0.5\% dependence on the choice of physics lists.  

Finally the importance of the low energy/low response particles near the fringe of the hadronic recoil to the response of the total recoil is studied by way of the topology correction.  
These low energy particles are found to bring the response of the recoil down approximately 5\% when compared to the response of the core, as reconstructed using an anti-$k_\mathrm{t}$ R=0.4 jet.  
This difference once again decreases with increasing jet size.  
This quantity is unaffected by the choice of physics lists above 30-40 GeV, with the low energy regime being affected by the difference in response of the two lists used changing the size of the impact of the jet reconstruction threshold.  

The dependence on the choice of physics lists for these two quantities can be added as a uncertainty on the MPF method for in-situ calibration.  
The uncertainty on these effects have previously been covered by adding the uncertainty from the so-called out of cone (OOC) correction for the $p_{\mathrm{T}}$ balance method, an unrelated quantity which was believed to be a conservative estimate on the uncertainty.  
Using the dependence of the measured total correction on the choice of physics lists as a measure of the uncertainty on this quantity in place of the OOC uncertainty leads to a reduction from 3\% to 0.5\% uncertainty in the lowest energy bin for the Z+jet analysis, with the uncertainty remaining approximately unchanged above 35 GeV.  
In the $\gamma$+jet channel after the rebinning described in Sec.~\ref{Sec:Systs} there is no depence of this total correction on the choice of physics lists.  
This is a recution from the OOC uncertainty which in 2016 was approximately 1\% below 50 GeV and 0.5\% in the 50-100 GeV range.  
The transition to this more accurate set of uncertainties will further reduce the overall uncertainty on the JES for jets below 100 GeV.  
 



